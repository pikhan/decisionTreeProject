\documentclass{article}
\begin{document}
    These classes will form the basis for our tree. Clearly, we need to denote
left and right children of each node in the tree, but we also need to
keep track of which samples are "in" each node during the model's training
so that we can form output labels and see what is going on during the debugging
process. Further, we need to make sure during training we do not split on
the same feature twice so we will keep track of un-split features as well
as which feature we choose to split on. These properties will all be None
upon the return of DT_train_binary save for feature_split as the rest are
only needed during training. In addition to this, the two node classes
also have a label prediction. The real node class has some additional structure
as the feature list is now much larger (we can split our tree on any inequality
on any feature), hence the need for feature_split, feature_value, and feature_sign.

\end{document}